\documentclass[paper=a4, fontsize=11pt]{article}

\usepackage{sectsty}
\allsectionsfont{\normalfont\scshape}

\usepackage{fancyhdr} 
\fancyhead{}
\fancyfoot[C]{} 
\fancyfoot[C]{} 
\fancyfoot[C]{\thepage} 
\renewcommand{\headrulewidth}{0pt} 
\renewcommand{\footrulewidth}{0pt} 
\setlength{\headheight}{13.6pt} 

\setlength{\parindent}{4em}
\setlength{\parskip}{1em}
\renewcommand{\baselinestretch}{1.5}

\usepackage{graphicx}
\graphicspath{ {images/} }

\usepackage{xepersian}
\settextfont[Path=fonts/]{Vazir.ttf}

\newcommand{\horrule}[1]{\rule{\linewidth}{#1}} 

\title{
\normalfont\normalsize
\includegraphics[scale=0.1]{aut}
\hspace{5cm}
\includegraphics[scale=0.1]{ceit} \\
\textsc دانشگاه صنعتی امیرکبیر \\
\textsc دانشکده مهندسی کامپیوتر و فناوری اطلاعات
\horrule{0.5pt} \\ [0.4cm]
\huge کاربرد سیستم‌های تشخیص چهره\\ در گوشی‌ها
\horrule{2pt} \\ [0.5cm]
}

\author{سید نوید کرمی‌نژاد \\ ۹۴۳۱۰۷۰}

\begin{document}

\maketitle
\thispagestyle{empty}
\newpage
\setcounter{page}{1}

\section*{\LARGE مقدمه}

\par
با توجه به استفاده روزافزون از گوشی‌های هوشمند و ضرورت حفاظت اطلاعات شخصی کاربران نیاز به وجود سیستمی قدرتمند جهت تضمین امنیت گوشی‌های هوشمند در برابر حمله هکرها و همچنین روش‌های رمزگشایی، که مشکل نگرانی کاربران از ربوده شدن اطلاعات را برطرف کند،‌ حس می‌شود. امروزه استفاده از گوشی‌های موبایل فراتر از برقراری تماس با سایرین رفته و بسیاری از افراد اطلاعات شخصی مانند رمزهای بانکی، عکس‌ها و یادداشت‌های شخصی، اطلاعات محل کار، شماره تماس اشخاص و ... را در گوشی‌های خود نگه‌داری می‌کنند و به همین جهت نگرانی آن‌ها از بابت درز اطلاعات به روش‌های مختلف افزایشیافته است.

\par
امروزه شرکت‌های ساخت گوشی‌های هوشمند به دنبال جلب رضایت بیشتر مشتریان خود هستند و همواره سعی در بهبود سیستم‌های حفاظتی خود دارند.

\par
در این پژوهش سعی داریم در مورد آخرین تکنولوژی به‌کارگرفته شده در این زمینه، یعنی سیستم تشخیص چهره، صحبت کرده و به معرفی این سیستم بپردازیم و عملکرد آن را مورد بررسی و ارزیابی قرار دهیم.

\newline

\section*{\LARGE مرور سوابق پیشین }

\par
به تازگی و از سال ۲۰۱۷ شرکت‌های بزرگ تولید گوشی‌های هوشمند نظیر اپل در مدل "آیفون ایکس" و سامسونگ در مدل "گلکسی اس ۸" خود این سیستم را قرار داده تا بدین صورت با بازخوردی که از خود مشتریان و کاربران به‌دست می‌آورد بتواند نواقص موجود را برطرف کنند و در آینده به سیستمی غیرقابل نفوذ دست‌یابند.

\section*{\LARGE طرح پیشنهادی }

\par
در این پروژه به دنبال آن هستیم تا پس از آشنایی با نحوه عملکرد سیستم تشخیص چهره و معرفی سیستم‌های امروزی به‌کار گرفته شده در گوشی‌های هوشمند و چگونگی کارکرد آن‌ها، میزان موفقیت این پروژه در کاربرد عمومی و در دنیای واقعی را مورد بررسی و ارزیابی قرار دهیم.

\section*{\LARGE محصولات طرح }

\par
خروجی های این طرح شامل موارد زیر خواهد بود:
\begin{itemize}
    \item	معرفی بهترین و موثرترین سیستم امنیتی موجود در حال حاضر
    \item 	پیشنهاد روشی جدید جهت بهبود عملکرد سیستم مربوطه
\end{itemize}

\section*{\LARGE مراحل انجام }
\begin{enumerate}
    \item شناسایی و تهیه منابع :
    \begin{itemize}
        \item آشنایی با نحوه عملکرد سیستم تشخیص چهره
        \item جست‌وجوی اینترنتی و کتابخانه‌ای برای مقالات و کتاب‌های مرتبط
        \item بررسی و انتخاب منابع معتبر و مناسب و حذف منابع نامربوط
    \end{itemize}
    \item تنظیم ساختار :
    \begin{itemize}
        \item تهیه بخش‌های مقدمه، محتوای اصلی، چکیده،‌ نتیجه‌گیری و منابع
        \item مشخص کردن ترتیب مباحث محتوا
        \item تهیه‌ی فهرست مباحث اصلی و فرعی
    \end{itemize}
    \newpage
    \item مطالعه و یادداشت برداری :
    \begin{itemize}
        \item مطالعه منابع اینترنتی
        \item مطالعه وب‌سایت‌های مفید درباره موضوع
        \item مطالعه کتاب‌های مرتبط
        \item مطالعه و یادداشت‌برداری از پایان‌نامه‌های مرتبط
        \item مشاهده‌ی ویدئوهای مربوط با موضوع در سایت YouTube
    \end{itemize}
    \item اجرای بخش عملی :
    \begin{itemize}
        \item آزمایش عملی سیستم‌های کنونی موجود در گوشی‌ها
        \item آزمایش سیستم جدید مطرح شده و بررسی نحوه عملکرد آن با سیستم‌های کنونی
    \end{itemize}
    \item تهیه گزارش نهایی :
    \begin{itemize}
        \item مستندسازی و نوشتن گزارش نهایی
        \item آمادگی جهت ارائه شفاهی
    \end{itemize}
\end{enumerate}
\newpage
\section*{\LARGE مراحل انجام }
\par
زمان بندی پروژه بر اساس برنامه زیر اجرا خواهد شد :
\newline
\newline
\includegraphics[]{Capture}
\end{document}